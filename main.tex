\documentclass{article}
\usepackage[utf8]{inputenc}
\usepackage[portuguese]{babel}
\usepackage{hyperref}
\usepackage{color}
\usepackage{float}
\newcommand{\blue}{\textcolor{blue}}
\usepackage{geometry}
\usepackage{array}
\geometry{a4paper, margin=2cm}
\usepackage{subfigure}
\usepackage{enumitem}
\usepackage{caption}
\title{Atividade de Sistemas Híbridos}
\author{Filipe Duarte e Gabriel G. Carvalho \\ \texttt{\{fcld,ggc5\}@cin.ufpe.br}}

\date{1 de Novembro de 2019}

\usepackage{natbib}
\usepackage{graphicx}

\begin{document}

\maketitle
\begin{abstract}
O objetivo dessa atividade é implementar e avaliar o desempennho um Sistema Híbirdo, conforme Zhang (2003), para realizar previsão do logaritmo da taxa de mortalidade da idade de 40 anos da população francesa.

\end{abstract}
\maketitle

\section{Introdução -- Sistemas Híbridos}
No contexto de {\it séries temporais}, doravanete chamaremos de {\it sistema híbrido} uma modelagem para a série temporal da forma 
\begin{equation}\label{hybriddef}
    Z_t = L_t + N_t
\end{equation}
na qual $L_t$ representa a componente com modelagem {\it linear}, enquanto $N_t$ representa a componente cuja modelagem é não linear.
Neste trabalho, usamos o método clássico {\bf ARIMA} para modelar a componente linear $L_t$. A componente não linear, $N_t$, foi modelada a usando Redes Neurais Artificiais, mais especificamente usamos a chamada Multilayer Perceptron (MLP). Para deixar o trabalho mais completo, faremos uma modelagem de $Z_t$ apenas usando {\bf ARIMA}, e outra modelagem apenas usando MLP.

\section{Descrição do Trabalho -- Metodologia}
A modelagem puramente usando {\bf ARIMA} , ou puramente usando uma MLP já foram trabalhadas. Portanto, aqui iremos descrever o procedimento usado para a previsão da série na modelagem híbrida. Dividiremos a série em treinamento e teste. 
\begin{enumerate}
    \item Usaremos uma modelagem {\bf ARIMA} no conjunto de treinamento, calculando o erro da modelagem;
    \item Treinamos a MLP no erro obtido ao modelar a série usando {\bf ARIMA};
    \item Realizar a previsão do modelo {\bf ARIMA}: $h=20$;
    \item Realizar a preivsão pela MLP com $20$ saídas.
\end{enumerate}
\section{Estudo da Série}
A série escolhida está no grão anual, e tem duração de 202 anos. 
\begin{figure}[H]
    \centering
    \includegraphics[scale=0.4]{mortalityrate40.png}
    \caption{Série da taxa de mortalidade - logarítmica}
\end{figure}
Como iremos aplicar uma janela com 2 lags passados e 19 anos para previsão, dividimos o conjunto de treinamento de $1816$ àc $1990$ e o conjunto de teste como sendo de $1991$ à $2010$. 
\begin{figure}[H]
    \centering
    \includegraphics[scale=0.4]{train_test.png}
    \caption{Divisão dos conjuntos de treinamento e teste}
\end{figure}
\subsection{Estatística descritiva}
Para o conjunto de treinamento temos:
\begin{enumerate}
    \item Amostras: $175$
    \item Média: $-4.896133$
    \item Desvio padrão: $0.619546$
    \item Valor mínimo: $-6.207136$
    \item Valor máximo: $-3.895854$
    \item $(25\%,50\%,75\%) = (-5.475952,-4.577069,-4.481008)$
\end{enumerate}
Para o conjunto de teste temos:
\begin{enumerate}
    \item Amostras: $20$
    \item Média: $-6.366835$
    \item Desvio padrão: $0.178995$
    \item Valor mínimo: $-6.687817$
    \item Valor máximo: $-6.123854$
    \item $(25\%,50\%,75\%) = (-6.506776,-6.355612,-6.183168)$
\end{enumerate}
Além dos dados estatísticos acima, para o conjunto de treinamento temos:
\begin{enumerate}
    \item ADF Statistic: $0.180194$
    \item p-value: $0.971129$
    \item Desvio padrão: $0.619546$
    \item Valores críticos:
    \begin{itemize}
        \item $1\%$: $-3.469$
        \item $5\%$: $-2.879$
        \item $10\%$: $-2.576$
    \end{itemize}
\end{enumerate}
Os resultados indicam que a série não é estacionária. Para concluir, vejamos os gráficos ACF e PACF do conjunto de treinamento:
\begin{figure}[H]
    \centering
    \includegraphics[scale=0.4]{acf.png}
    \caption{Autocorrelação}
\end{figure}
\begin{figure}[H]
    \centering
    \includegraphics[scale=0.4]{pacf.png}
    \caption{Autocorrelação Parcial}
\end{figure}


\section{Modelo ARIMA}

\section{Modelo MLP}

\section{ARIMA + MLP}

\section{Resultados}

\section{Conclusão}
O modelo ARIMA apresentou um RMSE, para os dados de teste, no valor aproximado de $0.1496$. Contudo, quando implementado o sistema híbrido a partir da soma do modelo ARIMA com o modelo MLP, o RMSE aumentou para o valor aproximado de $0.1572$.

Sendo assim, o Sistem Híbrido não apresentou melhores previsões do que o ARIMA. Duas possíveis justificativas podem explicar esse fato:
\begin{itemize}
    \item O erro de treinamento do ARIMA, que serviu para treinar o modelo MLP, seria uma série não autocorrelacionada.
    \item O Tamanho da janela de previsão ser de longo prazo.
\end{itemize}
Vale a pena investigar se outro modelo de machine learning, por exemplo um SVR, conseguiria mapear de forma mais adequada o erro gerado pelo modelo arima. Ademais, seria interessante implementar outra combinação de sistema híbrido, por meio de técnicas como processos gaussianos etc.

\end{document}
